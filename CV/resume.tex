\documentclass[10pt,a4paper]{article}
\usepackage{geometry}
\usepackage{xcolor}
\usepackage[scaled]{helvet}
\usepackage{fontawesome5}
\usepackage{hyperref}
\usepackage[none]{hyphenat}
\usepackage{graphicx}
\usepackage{enumitem}
\usepackage{fontspec}
\defaultfontfeatures{Ligatures=TeX}
\setmainfont{DejaVu Sans}
\setsansfont{DejaVu Sans}
\usepackage[russian,english]{babel}

\renewcommand{\familydefault}{\sfdefault}

\geometry{
  left=15pt, right=15pt,
  top=15pt, bottom=15pt
}

\begin{document}

\noindent
\begin{minipage}[t]{0.40\textwidth}
    \small
    \includegraphics[width=2.5cm]{photo.jpg}\\[1em]

    {\large\bfseries Персональная информация}\\[-0.2em]
    \rule{\linewidth}{0.4pt}\\[0.4em]
    \begin{tabular}{@{}rl}
        \faPhone\ & +7 963 752 92 18 \\
        \faEnvelope\ & \href{mailto:inizhankovskii@gmail.com}{inizhankovskii@gmail.com} \\
        \faMapMarker\ & Москва \\
    \end{tabular}

    \vspace{1em}
    {\large\bfseries Links}\\[-0.2em]
    \rule{\linewidth}{0.4pt}\\[0.4em]
    \begin{tabular}{@{}rl}
        \faTelegram\ & \href{https://t.me/alwaysontire}{@alwaysontire} \\
    \end{tabular}

    \vspace{1em}
    {\large\bfseries Навыки программирования}\\[-0.2em]
    \rule{\linewidth}{0.4pt}\\[0.4em]
    \begin{tabular}{@{}rl}
        \faCircleNotch\ & Python \\
        \faCircleNotch\ & C++ \\
        \faCircleNotch\ & Go \\
        \faCircleNotch\ & ML \\
        \faCircleNotch\ & Bash \\
    \end{tabular}

    \vspace{1em}
    {\large\bfseries Разговорные языки}\\[-0.2em]
    \rule{\linewidth}{0.4pt}\\[0.4em]
    \begin{tabular}{@{}rl}
        \faLanguage\ & Russian \\
        \faLanguage\ & English \\
    \end{tabular}
\end{minipage}%
\hfill%
\begin{minipage}[t]{0.58\textwidth}
    {\huge\bfseries Нижанковский Илья}\\[0.3em]
    {\large Data Scientist}

    \vspace{1em}
    Я математик и программист с прочной основой в алгоритмах, логике и решении задач. 
    Обладаю глубокими знаниями в области математики и компьютерных наук, развивал 
    аналитическое мышление и навыки программирования через практические проекты: 
    создание ML-модели на основе кластеризации, web-приложения на Streamlit для 
    прогноза погоды. Особое внимание уделяю глубинному обучению и хотел бы 
    связать с ним свою карьеру. Имею опыт IT-ассистента, что помогает понимать 
    корпоративную культуру и реалии работы в компании.

    \vspace{1.2em}
    {\large\bfseries Опыт работы}\\[-0.2em]
    \rule{\linewidth}{0.4pt}\\[0.4em]
    {\bfseries Стажировка} \hfill Июль 2023 – Апрель 2024\\
    \begin{itemize}[leftmargin=*]\itemsep0pt
        \item Помогал в IT-отделе «Горизонт Покрытий»: писал инструкции, создал Telegram-бота и интегрировал его с Zabbix для мониторинга.
    \end{itemize}

    \vspace{0.8em}
    {\large\bfseries Образование}\\[-0.2em]
    \rule{\linewidth}{0.4pt}\\[0.6em]

    {\bfseries Высшая Школа Экономики (бакалавриат)} \hfill Сентябрь 2024 – Июнь 2025\\
    Направление: Прикладная математика и информатика\\
    Программа: Компьютерные науки и анализ данных

    \vspace{0.8em}
    {\bfseries Павловская гимназия (среднее образование)} \hfill Сентябрь 2011 – Июнь 2022\\
    Профиль: Математика, информатика, физика

    \vspace{1em}
    {\large\bfseries Увлечения}\\[-0.2em]
    \rule{\linewidth}{0.4pt}\\[0.4em]
    \begin{itemize}[leftmargin=*]\itemsep0pt
        \item Триатлон: регулярно тренируюсь, участвовал в соревнованиях
        \item Машинное обучение: реализую личные исследования и pet-проекты
    \end{itemize}
\end{minipage}

\end{document}
